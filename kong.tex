\documentclass[a5paper]{ctexbook}
\usepackage{geometry}
\usepackage{hyperref}
\geometry{a5paper}
\title{《论语》整理}
\author{安梯西登}
\date{}
\begin{document}
    \maketitle

    \tableofcontents

    \chapter{学}

    子曰:“学而时习之,不亦说乎?”

    子曰:“吾与回言终日,不违,如愚。退而省其私,亦足以发,回也不愚。”

    子曰:“学而不思则罔,思而不学则殆。”

    子曰:“攻乎异端,斯害也已。”

    子曰:“三人行,必有我师焉。择其善者而从之,其不善者而改之。”

    子曰:“学如不及,犹恐失之。”

    \chapter{朋友}

    子曰:“有朋自远方来,不亦乐乎?”

    “唐棣之华,偏其反而。岂不尔思?室是远而〔1〕。”子曰:“未之思也,夫何远之有?”

    \chapter{君子}
    
    子曰:“人不知而不愠,不亦君子乎?”

    子曰:“君子不重则不威,学则不固。主忠信。无友不如己者。过,则勿惮改。”

    子曰:“君子食无求饱,居无求安,敏于事而慎于言,就有道而正焉,可谓好学也已。”

    子曰:“君子不器。”
    
    子贡问君子。子曰:“先行其言而后从之。”
    
    子曰:“君子周而不比,小人比而不周。”

    子曰:“君子无所争。必也射乎!揖让而升,下而饮。其争也君子。”

    子曰:“君子之于天下也,无适也,无莫也,义之与比。”

    子曰:“君子怀德,小人怀土;君子怀刑,小人怀惠。”

    子曰:“君子喻于义,小人喻于利。”

    子曰:“君子欲讷于言而敏于行。”

    子谓子产:“有君子之道四焉:其行己也恭,其事上也敬,其养民也惠,其使民也义。”

    子华使于齐,冉子为其母请粟。子曰:“与之釜。”请益。曰:“与之庾。”冉子与之粟五秉。子曰:“赤之适齐也,乘肥马,衣轻裘。吾闻之也:君子周急不继富。”

    子谓子夏曰:“女为君子儒,无为小人儒!”

    子曰:“质胜文则野,文胜质则史。文质彬彬,然后君子。”

    宰我问曰:“仁者,虽告之曰‘井有仁焉’,其从之也?”子曰:“何为其然也?君子可逝也,不可陷也;可欺也,不可罔也。”

    子曰:“君子博学于文,约之以礼,亦可以弗畔矣夫。”

    子曰:“君子坦荡荡,小人长戚戚。”

    \chapter{仁}

    子曰:“巧言令色,鲜矣仁!”

    子曰:“人而不仁,如礼何?人而不仁,如乐何?”

    子夏问曰:“‘巧笑倩兮,美目盼兮,素以为绚兮。’何谓也?”子曰:“绘事后素。”曰:“礼后乎?”子曰:“起予者商也!始可与言《诗》已矣。”

    子曰:“里仁为美。择不处仁,焉得知?”

    子曰:“不仁者不可以久处约,不可以长处乐。仁者安仁,知者利仁。”

    子曰:“唯仁者能好人,能恶人。”

    子曰:“苟志于仁矣,无恶也。”

    子曰:“富与贵,是人之所欲也;不以其道得之,不处也。贫与贱,是人之所恶也;不以其道得之,不去也。君子去仁,恶乎成名?君子无终食之间违仁,造次必于是,颠沛必于是。”

    子曰:“人之过也,各于其党。观过,斯知仁矣。”

    子曰:“我未见好仁者,恶不仁者。好仁者,无以尚之;恶不仁者,其为仁矣,不使不仁者加乎其身。有能一日用其力于仁矣乎?我未见力不足者。盖有之矣,我未之见也。”

    子张问曰:“令尹子文三仕为令尹,无喜色;三已之,无愠色。旧令尹之政,必以告新令尹。何如?”子曰:“忠矣。”曰:“仁矣乎?”曰:“未知,焉得仁?”“崔子弑齐君,陈文子有马十乘,弃而违之。至于他邦,则曰:‘犹吾大夫崔子也。’违之。之一邦,则又曰:‘犹吾大夫崔子也。’违之。何如?”子曰:“清矣。”曰:“仁矣乎?”曰:“未知,焉得仁?”

    樊迟问仁。子曰:“仁者先难而后获,可谓仁矣。”

    子曰:“仁者乐山。仁者静。仁者寿。”

    子贡曰:“如有博施于民而能济众,何如?可谓仁乎?”子曰:“何事于仁!必也圣乎!尧、舜其犹病诸!夫仁者,己欲立而立人,己欲达而达人。能近取譬,可谓仁之方也已。”

    子曰:“仁者不忧。”

    \chapter{中庸}

    子曰:“中庸之为德也,其至矣乎!民鲜久矣。”

    \chapter{诗}

    子曰:“诗三百,一言以蔽之,曰:‘思无邪。’”

    子曰:“《关雎》乐而不淫,哀而不伤。”

    \chapter{礼}

    子张问:“十世可知也?”子曰:“殷因于夏礼,所损益,可知也;周因于殷礼,所损益,可知也。其或继周者,虽百世,可知也。”

    子曰:“周监于二代,郁郁乎文哉!吾从周。”

    孔子谓季氏,“八佾舞于庭,是可忍也,孰不可忍也?”

    三家者以《雍》彻。子曰:“‘相维辟公,天子穆穆’,奚取于三家之堂?”

    林放问礼之本。子曰:“大哉问!礼,与其奢也,宁俭。丧,与其易也,宁戚。”

    子曰:“夷狄之有君,不如诸夏之亡也。”

    季氏旅于泰山。子谓冉有曰:“女弗能救与?”对曰:“不能。”子曰:“呜呼!曾谓泰山不如林放乎?”

    子曰:“夏礼,吾能言之,杞不足征也;殷礼,吾能言之,宋不足征也。文献不足故也。足,则吾能征之矣。”

    子曰:“禘自既灌而往者,吾不欲观之矣。”

    子入大庙,每事问。或曰:“孰谓鄹人之子知礼乎?入大庙,每事问。”子闻之,曰:“是礼也。”

    子贡欲去告朔之饩羊。子曰:“赐也!尔爱其羊,我爱其礼。”

    “然则管仲知礼乎?”曰:“邦君树塞门,管氏亦树塞门。邦君为两君之好,有反坫,管氏亦有反坫。管氏而知礼,孰不知礼?”

    子曰:“麻冕,礼也。今也纯〔1〕,俭,吾从众。拜下〔2〕,礼也。今拜乎上,泰也。虽违众,吾从下。”

    \chapter{乐}

    子语鲁大师乐,曰:“乐其可知也:始作,翕如也;从之,纯如也,皦如也,绎如也,以成。”

    子谓《韶》,“尽美矣,又尽善也”。谓《武》,“尽美矣,未尽善也”。

    子在齐闻《韶》,三月不知肉味,曰:“不图为乐之至于斯也。”

    子曰:“师挚之始〔1〕,《关雎》之乱〔2〕,洋洋乎盈耳哉〔3〕!”

    子曰:“吾自卫反鲁〔1〕,然后乐正,《雅》《颂》〔2〕各得其所。”

    \chapter{智}

    子曰:“由!诲女知之乎!知之为知之,不知为不知,是知也。”

    樊迟问知。子曰:“务民之义,敬鬼神而远之,可谓知矣。”
    
    子曰:“知者乐水。知者动。知者乐。”

    子曰:“中人以上,可以语上也;中人以下,不可以语上也。”

    子曰:“知者不惑。”

    \chapter{鬼神}

    或问禘之说。子曰:“不知也。知其说者之于天下也,其如示诸斯乎!”指其掌。

    祭如在,祭神如神在。子曰:“吾不与祭,如不祭。”

    王孙贾问曰:“与其媚于奥,宁媚于灶,何谓也?”子曰:“不然。获罪于天,无所祷也。”

    \chapter{孝}

    子曰:“父在,观其志;父没,观其行;三年无改于父之道,可谓孝矣。”

    孟懿子问孝。子曰:“无违。”樊迟御,子告之曰:“孟孙问孝于我,我对曰,无违。”樊迟曰:“何谓也?”子曰:“生,事之以礼;死,葬之以礼,祭之以礼。”

    孟武伯问孝。子曰:“父母唯其疾之忧。”

    子游问孝。子曰:“今之孝者,是谓能养。至于犬马,皆能有养。不敬,何以别乎?”

    子夏问孝。子曰:“色难。有事,弟子服其劳;有酒食,先生馔,曾是以为孝乎?”

    子曰:“事父母几谏,见志不从,又敬不违,劳而不怨。”

    子曰:“父母在,不远游,游必有方。”

    子曰:“父母之年,不可不知也。一则以喜,一则以惧。”

    \chapter{信}

    子曰:“人而无信,不知其可也。大车无𫐐,小车无𫐄,其何以行之哉?”

    \chapter{俭}

    或曰:“管仲俭乎?”曰:“管氏有三归,官事不摄,焉得俭?”

    \chapter{治国}

    子曰:“道千乘之国,敬事而信,节用而爱人,使民以时。”

    子曰:“为政以德,譬如北辰居其所而众星共之。”

    子曰:“道之以政,齐之以刑,民免而无耻。道之以德,齐之以礼,有耻且格。”

    哀公问曰:“何为则民服?”孔子对曰:“举直错诸枉,则民服;举枉错诸直,则民不服。”

    季康子问:“使民敬、忠以劝,如之何?”子曰:“临之以庄,则敬;孝慈,则忠;举善而教不能,则劝。”

    子曰:“能以礼让为国乎?何有?不能以礼让为国,如礼何?”

    子曰:“民可使由之,不可使知之。”

    \chapter{修身}

    子曰:“弟子入则孝,出则弟,谨而信,泛爱众而亲仁。行有余力,则以学文。”

    子贡曰:“贫而无谄,富而无骄,何如?”子曰:“可也。未若贫而乐,富而好礼者也。”子贡曰:“《诗》云:‘如切如磋,如琢如磨。’其斯之谓与?”子曰:“赐也,始可与言《诗》已矣,告诸往而知来者。”

    子曰:“见贤思齐焉,见不贤而内自省也。”

    子曰:“兴于诗,立于礼,成于乐。”

    子曰:“如有周公之才之美,使骄且吝,其余不足观也已。”

    子曰:“狂而不直,侗而不愿〔1〕,悾悾而不信〔2〕,吾不知之矣。”

    \chapter{利}

    子曰:“富而可求也,虽执鞭之士〔1〕,吾亦为之。如不可求,从吾所好。”

    子曰:“饭疏食饮水〔1〕,曲肱而枕之〔2〕,乐亦在其中矣。不义而富且贵,于我如浮云。”

    \chapter{教}

    子曰:“温故而知新,可以为师矣。”

    子曰:“自行束脩以上〔1〕,吾未尝无诲焉。”

    子曰:“不愤不启〔1〕,不悱不发〔2〕。举一隅不以三隅反〔3〕,则不复也。

    子以四教:文、行、忠、信。

    \chapter{行}

    子之燕居〔1〕,申申如也,夭夭如也〔2〕。

    子食于有丧者之侧,未尝饱也。

    子于是日哭,则不歌。

    子之所慎:齐〔1〕,战,疾。

    子所雅言〔1〕,《诗》、《书》、执礼,皆雅言也。

    子钓而不纲〔1〕,弋不射宿〔2〕。

    子与人歌而善,必使反之,而后和之。

    子温而厉,威而不猛,恭而安。

    子罕言利与命与仁〔1〕。

    子绝四:毋意,毋必,毋固,毋我。

    子见齐衰者〔1〕、冕衣裳者与瞽者〔2〕,见之,虽少,必作〔3〕,过之,必趋〔4〕。

    孔子于乡党,恂恂如也〔1〕,似不能言者。其在宗庙朝廷,便便言〔2〕,唯谨尔。

    朝,与下大夫言〔1〕,侃侃如也;与上大夫言,訚訚如也〔2〕。君在,踧踖如也〔3〕,与与如也〔4〕。

    君召使摈〔1〕,色勃如也〔2〕,足躩如也〔3〕。揖所与立,左右手,衣前后,襜如也〔4〕。趋进,翼如也。宾退,必复命曰:“宾不顾矣〔5〕。”

    入公门,鞠躬如也〔1〕,如不容。立不中门,行不履阈〔2〕。过位,色勃如也,足躩如也,其言似不足者。摄齐升堂〔3〕,鞠躬如也,屏气似不息者。出,降一等〔4〕,逞颜色〔5〕,怡怡如也。没阶,趋进,翼如也。复其位,踧踖如也。

    执圭〔1〕,鞠躬如也,如不胜。上如揖,下如授。勃如战色,足蹜蹜如有循〔2〕。享礼〔3〕,有容色。私觌〔4〕,愉愉如也。

    君子不以绀𮉪饰〔1〕,红紫不以为亵服〔2〕。当暑,袗𫄨绤〔3〕,必表而出之。缁衣〔4〕,羔裘〔5〕;素衣,麑裘〔6〕;黄衣,狐裘。亵裘长,短右袂〔7〕。必有寝衣〔8〕,长一身有半。狐貉之厚以居〔9〕。去丧,无所不佩。非帷裳〔10〕,必杀之〔11〕。羔裘玄冠不以吊〔12〕。吉月〔13〕,必朝服而朝。

    齐,必有明衣〔1〕,布。齐必变食〔2〕,居必迁坐〔3〕。

    食不厌精,脍不厌细〔1〕。食𮩞而𮩝〔2〕,鱼馁而肉败〔3〕,不食。色恶,不食。臭恶〔4〕,不食。失饪,不食。不时,不食。割不正,不食。不得其酱,不食。肉虽多,不使胜食气〔5〕。惟酒无量,不及乱。沽酒市脯不食。不撤姜食,不多食。

    祭于公,不宿肉〔1〕。祭肉不出三日〔2〕。出三日,不食之矣。【
    
    食不语,寝不言。
    
    虽疏食菜羹〔1〕,必祭〔2〕,必齐如也〔3〕。

    席不正,不坐。
    
    乡人饮酒〔1〕,杖者出〔2〕,斯出矣。
    
    乡人儺〔1〕,朝服而立于阼阶〔2〕。
    
    问人于他邦〔1〕,再拜而送之〔2〕。

    君赐食,必正席先尝之。君赐腥〔1〕,必熟而荐之〔2〕。君赐生,必畜之。侍食于君,君祭,先饭〔3〕。

    疾,君视之,东首〔1〕,加朝服,拖绅〔2〕。
    
    君命召,不俟驾行矣。

    朋友死,无所归,曰:“于我殡〔1〕。”
    
    朋友之馈,虽车马,非祭肉,不拜。
    
    寝不尸,居不客〔1〕。

    见齐衰者,虽狎必变〔1〕。见冕者与瞽者,虽亵必以貌〔2〕。凶服者式之〔3〕。式负版者〔4〕。有盛馔,必变色而作〔5〕。迅雷风烈必变。

    升车,必正立,执绥〔1〕。车中不内顾,不疾言,不亲指。

    \chapter{谓己}

    子曰:“吾十有五而志于学,三十而立,四十而不惑,五十而知天命,六十而耳顺,七十而从心所欲,不逾矩。”

    子曰:“十室之邑,必有忠信如丘者焉,不如丘之好学也。”

    子曰:“述而不作〔1〕,信而好古,窃比于我老彭〔2〕。”

    子曰:“默而识之〔1〕,学而不厌,诲人不倦,何有于我哉?”

    子曰:“甚矣吾衰也!久矣吾不复梦见周公〔1〕!”

    子曰:“加我数年,五十以学《易》〔1〕,可以无大过矣。”

    叶公问孔子于子路〔1〕,子路不对。子曰:“女奚不曰:‘其为人也,发愤忘食,乐以忘忧,不知老之将至云尔。’”

    子曰:“我非生而知之者,好古,敏以求之者也。”

    子曰:“天生德于予,桓魋其如予何〔1〕?”

    子曰:“二三子以我为隐乎?吾无隐乎尔。吾无行而不与二三子者〔1〕,是丘也。”

    子曰:“盖有不知而作之者,我无是也。多闻,择其善者而从之;多见而识之;知之次也〔1〕。”

    子曰:“仁远乎哉?我欲仁,斯仁至矣。”

    子曰:“文,莫吾犹人也〔1〕。躬行君子,则吾未之有得。”

    子曰:“若圣与仁,则吾岂敢?抑为之不厌,诲人不倦,则可谓云尔已矣。”

    子畏于匡〔1〕。曰:“文王既没,文不在兹乎?天之将丧斯文也,后死者不得与于斯文也〔2〕。天之未丧斯文也,匡人其如予何?”

    子曰:“吾有知乎哉?无知也。有鄙夫问于我,空空如也。我叩其两端而竭焉〔1〕。”

    子曰:“凤鸟不至〔1〕,河不出图〔2〕,吾已矣夫〔3〕!”

    子贡曰:“有美玉于斯,韫椟而藏诸〔1〕?求善贾而沽诸〔2〕?”子曰:“沽之哉!沽之哉!我待贾者也。”

    子欲居九夷〔1〕。或曰:“陋,如之何?”子曰:“君子居之,何陋之有?”

    子曰:“出则事公卿,入则事父兄,丧事不敢不勉,不为酒困,何有于我哉?”

    \chapter{谓人}

    子在陈,曰:“归与!归与!吾党之小子狂简,斐然成章,不知所以裁之。”

    子曰:“回也,其心三月不违仁,其余则日月至焉而已矣。”

    哀公问:“弟子孰为好学?”孔子对曰:“有颜回者好学,不迁怒,不贰过。不幸短命死矣。今也则亡,未闻好学者也。”

    子曰:“贤哉,回也!一箪食,一瓢饮,在陋巷,人不堪其忧,回也不改其乐。贤哉,回也!”

    子曰:“语之而不惰者,其回也与!”

    子谓颜渊,曰:“惜乎!吾见其进也,未见其止也。”

    子谓公冶长:“可妻也。虽在缧绁之中,非其罪也。”以其子妻之。

    子谓南容:“邦有道,不废;邦无道,免于刑戮。”以其兄之子妻之。

    子谓子贱:“君子哉若人!鲁无君子者,斯焉取斯?”

    子贡问曰:“赐也何如?”子曰:“女,器也。”曰:“何器也?”曰:“瑚琏也。”

    子曰:“道不行,乘桴浮于海。从我者,其由与?”子路闻之喜。子曰:“由也好勇过我,无所取材。”

    子曰:“衣敝缊袍〔1〕,与衣狐貉者立〔2〕,而不耻者,其由也与?‘不忮不求,何用不臧〔3〕?’”子路终身诵之。子曰:“是道也,何足以臧?”

    孟武伯问:“子路仁乎?”子曰:“不知也。”又问。子曰:“由也,千乘之国,可使治其赋也,不知其仁也。”
    
    孟武伯问:“求也何如?”子曰:“求也,千室之邑,百乘之家,可使为之宰也,不知其仁也。”
    
    孟武伯问:“赤也何如?”子曰:“赤也,束带立于朝,可使与宾客言也,不知其仁也。”

    子谓子贡曰:“女与回也孰愈?”对曰:“赐也何敢望回?回也闻一以知十,赐也闻一以知二。”子曰:“弗如也,吾与女弗如也。”

    宰予昼寝。子曰:“朽木不可雕也,粪土之墙不可杇也。于予与何诛?”子曰:“始吾于人也,听其言而信其行;今吾于人也,听其言而观其行。于予与改是。”

    子曰:“吾未见刚者。”或对曰:“申枨。”子曰:“枨也欲,焉得刚?”

    子贡曰:“我不欲人之加诸我也,吾亦欲无加诸人。”子曰:“赐也,非尔所及也。”

    季康子问:“仲由可使从政也与?”子曰:“由也果,于从政乎何有?”
    
    季康子问:“赐也可使从政也与?”子曰:“赐也达,于从政乎何有?”
    
    季康子问:“求也可使从政也与?”子曰:“求也艺,于从政乎何有?”

    伯牛有疾,子问之,自牖执其手,曰:“亡之,命矣夫!斯人也而有斯疾也!斯人也而有斯疾也!”

    子谓仲弓,曰:“犁牛之子骍且角,虽欲勿用,山川其舍诸?”

    子曰:“孟之反不伐,奔而殿,将入门,策其马,曰:‘非敢后也,马不进也。’”

    子曰:“晏平仲善与人交,久而敬之。”

    子曰:“臧文仲居蔡,山节藻棁,何如其知也?”

    子曰:“宁武子,邦有道,则知;邦无道,则愚。其知可及也,其愚不可及也。”

    子曰:“伯夷、叔齐不念旧恶,怨是用希。”

    子曰:“雍也可使南面。”

    仲弓问子桑伯子。子曰:“可也,简。”仲弓曰:“居敬而行简,以临其民,不亦可乎?居简而行简,无乃大简乎?”子曰:“雍之言然。”

    子曰:“管仲之器小哉!”

    子曰:“孰谓微生高直?或乞醯焉,乞诸其邻而与之。”

    子曰:“泰伯〔1〕,其可谓至德也已矣。三以天下让〔2〕,民无得而称焉。”

    子曰:“巍巍乎!舜、禹之有天下也而不与焉〔1〕。”

    子曰:“大哉尧之为君也!巍巍乎!唯天为大,唯尧则之〔1〕。荡荡乎!民无能名焉。巍巍乎其有成功也,焕乎其有文章〔2〕!”

    子曰:“禹,吾无间然矣〔1〕。菲饮食而致孝乎鬼神〔2〕,恶衣服而致美乎黻冕〔3〕,卑宫室而尽力乎沟洫〔4〕。禹,吾无间然矣。”

    \chapter{杂}

    色斯举矣〔1〕,翔而后集〔2〕。曰:“山梁雌雉〔3〕,时哉时哉!”子路共之〔4〕,三嗅而作〔5〕。

    子曰:“可与共学,未可与适道〔1〕;可与适道,未可与立;可与立,未可与权〔2〕。”

    康子馈药,拜而受之。曰:“丘未达〔1〕,不敢尝。”
    
    厩焚〔1〕。子退朝,曰:“伤人乎?”不问马。

    子曰:“勇者不惧。”

    子曰:“岁寒,然后知松柏之后雕也〔1〕。”

    子曰:“三军可夺帅也,匹夫不可夺志也〔1〕。”

    子曰:“法语之言〔1〕,能无从乎?改之为贵。巽与之言〔2〕,能无说乎?绎之为贵〔3〕。说而不绎,从而不改,吾末如之何也已矣。”

    子曰:“后生可畏,焉知来者之不如今也?四十、五十而无闻焉,斯亦不足畏也已。”

    子曰:“苗而不秀者有矣夫〔1〕!秀而不实者有矣夫!”

    子曰:“譬如为山,未成一篑〔1〕,止,吾止也。譬如平地,虽覆一篑,进,吾往也。”

    子曰:“吾未见好德如好色者也。”

    子在川上,曰:“逝者如斯夫!不舍昼夜〔1〕。”

    子疾病,子路使门人为臣〔1〕。病间〔2〕,曰:“久矣哉,由之行诈也!无臣而为有臣。吾谁欺?欺天乎?且予与其死于臣之手也,无宁死于二三子之手乎!且予纵不得大葬,予死于道路乎?”

    牢曰〔1〕:“子云:‘吾不试〔2〕,故艺。’”

    太宰问于子贡曰〔1〕:“夫子圣者与?何其多能也?”子贡曰:“固天纵之将圣〔2〕,又多能也。”子闻之,曰:“太宰知我乎!吾少也贱,故多能鄙事。君子多乎哉?不多也。”

    达巷党人曰〔1〕:“大哉孔子!博学而无所成名。”子闻之,谓门弟子曰:“吾何执?执御乎?执射乎?吾执御矣。”

    舜有臣五人而天下治。武王曰:“予有乱臣十人〔1〕。”孔子曰:“才难,不其然乎?唐、虞之际〔2〕,于斯为盛〔3〕。有妇人焉,九人而已。三分天下有其二〔4〕,以服事殷。周之德,其可谓至德也已矣。”

    子曰:“不在其位,不谋其政。”

    子曰:“三年学,不至于谷〔1〕,不易得也。”

    子曰:“笃信好学,守死善道。危邦不入,乱邦不居。天下有道则见〔1〕,无道则隐。邦有道,贫且贱焉,耻也;邦无道,富且贵焉,耻也。”

    子曰:“好勇疾贫〔1〕,乱也。人而不仁,疾之已甚,乱也。”

    子曰:“奢则不孙〔1〕,俭则固〔2〕。与其不孙也,宁固。”

    子疾病〔1〕,子路请祷。子曰:“有诸?”子路对曰:“有之。《诔》曰〔2〕:‘祷尔于上下神祇〔3〕。’”子曰:“丘之祷久矣。”

    子曰:“恭而无礼则劳,慎而无礼则葸〔1〕,勇而无礼则乱,直而无礼则绞〔2〕。君子笃于亲,则民兴于仁;故旧不遗,则民不偷〔3〕。”
    
    公西华曰:“子为之不厌,诲人不倦,正唯弟子不能学也。”

    子曰:“不患人之不己知,患不知人也。”

    仪封人请见,曰:“君子之至于斯也,吾未尝不得见也。”从者见之。出曰:“二三子何患于丧乎?天下之无道也久矣,天将以夫子为木铎。”

    原思为之宰,与之粟九百,辞。子曰:“毋!以与尔邻里乡党乎!”

    子曰:“视其所以,观其所由,察其所安。人焉廋哉?人焉廋哉?”

    

    子曰:“德之不修,学之不讲,闻义不能徙〔1〕,不善不能改,是吾忧也。”

    子张学干禄。子曰:“多闻阙疑,慎言其余,则寡尤;多见阙殆,慎行其余,则寡悔。言寡尤,行寡悔,禄在其中矣。”

    或谓孔子曰:“子奚不为政?”子曰:“《书》云:‘孝乎惟孝,友于兄弟,施于有政。’是亦为政,奚其为为政?”

    子曰:“非其鬼而祭之,谄也。见义不为,无勇也。”

    子曰:“射不主皮,为力不同科,古之道也。”

    子曰:“事君尽礼,人以为谄也。”

    定公问:“君使臣,臣事君,如之何?”孔子对曰:“君使臣以礼,臣事君以忠。”

    或曰:“雍也仁而不佞。”子曰:“焉用佞?御人以口给,屡憎于人。不知其仁,焉用佞?”

    哀公问社于宰我。宰我对曰:“夏后氏以松,殷人以柏,周人以栗,曰使民战栗。”子闻之,曰:“成事不说,遂事不谏,既往不咎。”

    子使漆雕开仕。对曰:“吾斯之未能信。”子说。
    
    子曰:“朝闻道,夕死可矣。”
    
    子曰:“士志于道,而耻恶衣恶食者,未足与议也。”

    子贡问曰:“孔文子何以谓之‘文’也?”子曰:“敏而好学,不耻下问,是以谓之‘文’也。”

    子曰:“不患无位,患所以立。不患莫己知,求为可知也。”

    

    子曰:“古者言之不出,耻躬之不逮也。”

    季文子三思而后行。子闻之,曰:“再,斯可矣。”
    
    子曰:“以约失之者鲜矣。”

    子曰:“德不孤,必有邻。”

    子曰:“巧言、令色、足恭,左丘明耻之,丘亦耻之。匿怨而友其人,左丘明耻之,丘亦耻之。”

    颜渊、季路侍。子曰:“盍各言尔志?”子路曰:“愿车马衣轻裘与朋友共敝之而无憾。”颜渊曰:“愿无伐善,无施劳。”子路曰:“愿闻子之志。”子曰:“老者安之,朋友信之,少者怀之。”

    子曰:“已矣乎!吾未见能见其过而内自讼者也。”

    季氏使闵子骞为费宰。闵子骞曰:“善为我辞焉!如有复我者,则吾必在汶上矣。”

    冉求曰:“非不说子之道,力不足也。”子曰:“力不足者,中道而废。今女画。”

    子曰:“不有祝𬶍之佞,而有宋朝之美,难乎免于今之世矣。”

    子曰:“谁能出不由户?何莫由斯道也?”

    子曰:“人之生也直,罔之生也幸而免。”

    子曰:“知之者不如好之者,好之者不如乐之者。”

    

    子曰:“志于道,据于德,依于仁,游于艺〔1〕。”

    子曰:“齐一变,至于鲁;鲁一变,至于道。”

    子曰:“觚不觚,觚哉!觚哉!”

    子见南子,子路不说。夫子矢之曰:“予所否者,天厌之!天厌之!”

    子谓颜渊曰:“用之则行,舍之则藏〔1〕,惟我与尔有是夫。”子路曰:“子行三军〔2〕,则谁与〔3〕?”子曰:“暴虎冯河〔4〕,死而无悔者,吾不与也。必也临事而惧,好谋而成者也。”

    冉有曰:“夫子为卫君乎〔1〕?”子贡曰:“诺,吾将问之。”入,曰:“伯夷、叔齐何人也〔2〕?”曰:“古之贤人也。”曰:“怨乎?”曰:“求仁而得仁,又何怨?”出,曰:“夫子不为也。”

    子曰:“圣人,吾不得而见之矣;得见君子者,斯可矣。”子曰:“善人,吾不得而见之矣;得见有恒者,斯可矣。亡而为有,虚而为盈,约而为泰〔1〕,难乎有恒矣。”

    互乡难与言〔1〕,童子见,门人惑。子曰:“与其进也〔2〕,不与其退也,唯何甚?人洁己以进,与其洁也,不保其往也。”

    陈司败问〔1〕:“昭公知礼乎〔2〕?”孔子曰:“知礼。”孔子退,揖巫马期而进之〔3〕,曰:“吾闻君子不党〔4〕,君子亦党乎?君取于吴〔5〕,为同姓〔6〕,谓之吴孟子〔7〕。君而知礼,孰不知礼?”巫马期以告。子曰:“丘也幸,苟有过,人必知之。”

    \chapter{子张}

    子张曰:“士见危致命,见得思义,祭思敬,丧思哀,其可已矣。”
    
    子张曰:“执德不弘,信道不笃,焉能为有?焉能为亡?”
    
    子夏之门人问交于子张。子张曰:“子夏云何?”对曰:“子夏曰:‘可者与之,其不可者拒之。’”子张曰:“异乎吾所闻。君子尊贤而容众,嘉善而矜不能。我之大贤与,于人何所不容?我之不贤与,人将拒我,如之何其拒人也?”

    \chapter{子夏}

    子夏曰:“贤贤易色;事父母,能竭其力;事君,能致其身;与朋友交,言而有信。虽曰未学,吾必谓之学矣。”

    樊迟退,见子夏曰:“乡也吾见于夫子而问知,子曰:‘举直错诸枉,能使枉者直。’何谓也?”子夏曰:“富哉言乎!舜有天下,选于众,举皋陶,不仁者远矣。汤有天下,选于众,举伊尹,不仁者远矣。”

    子夏曰:“虽小道,必有可观者焉,致远恐泥,是以君子不为也。”
    
    子夏曰:“日知其所亡,月无忘其所能,可谓好学也已矣。”
    
    子夏曰:“博学而笃志,切问而近思,仁在其中矣。”

    子夏曰:“百工居肆以成其事,君子学以致其道。”
    
    子夏曰:“小人之过也必文。”
    
    子夏曰:“君子有三变:望之俨然,即之也温,听其言也厉。”

    子夏曰:“君子信而后劳其民;未信,则以为厉己也。信而后谏;未信,则以为谤己也。”
    
    子夏曰:“大德不逾闲,小德出入可也。”
    
    子游曰:“子夏之门人小子,当洒扫应对进退,则可矣,抑末也。本之则无,如之何?”子夏闻之,曰:“噫!言游过矣!君子之道,孰先传焉?孰后倦焉?譬诸草木,区以别矣。君子之道,焉可诬也?有始有卒者,其惟圣人乎!”

    子夏曰:“仕而优则学,学而优则仕。”

    \chapter{子游}

    子游曰:“事君数,斯辱矣。朋友数,斯疏矣。”

    子游曰:“丧致乎哀而止。”
    
    子游曰:“吾友张也为难能也,然而未仁。”

    子游为武城宰。子曰:“女得人焉尔乎?”曰:“有澹台灭明者,行不由径,非公事,未尝至于偃之室也。”

    \chapter{曾子}

    曾子曰:“吾日三省吾身:为人谋而不忠乎?与朋友交而不信乎?传不习乎?”

    曾子曰:“慎终追远,民德归厚矣。”

    子曰:“参乎!吾道一以贯之。”曾子曰:“唯。”子出,门人问曰:“何谓也?”曾子曰:“夫子之道,忠恕而已矣。”

    曾子有疾,召门弟子曰:“启予足!启予手!《诗》云:‘战战兢兢,如临深渊,如履薄冰。’而今而后,吾知免夫!小子!”

    曾子有疾,孟敬子问之。曾子言曰:“鸟之将死,其鸣也哀。人之将死,其言也善。君子所贵乎道者三:动容貌,斯远暴慢矣;正颜色,斯近信矣;出辞气,斯远鄙倍矣。笾豆之事,则有司存。”

    曾子曰:“以能问于不能,以多问于寡;有若无,实若虚,犯而不校。昔者吾友尝从事于斯矣。”
    
    曾子曰:“可以托六尺之孤,可以寄百里之命,临大节而不可夺也,君子人与?君子人也。”
    
    曾子曰:“士不可以不弘毅,任重而道远。仁以为己任,不亦重乎?死而后已,不亦远乎?”

    曾子曰:“君子以文会友,以友辅仁。”
    
    曾子曰:“君子思不出其位。”

    曾子曰:“堂堂乎张也,难于并为仁矣。”
    
    曾子曰:“吾闻诸夫子:人未有自致者也,必也亲丧乎!”
    
    曾子曰:“吾闻诸夫子:孟庄子之孝也,其他可能也,其不改父之臣与父之政,是难能也。”

    孟氏使阳肤为士师,问于曾子。曾子曰:“上失其道,民散久矣。如得其情,则哀矜而勿喜!”

    \chapter{子贡}

    子禽问于子贡曰:“夫子至于是邦也,必闻其政。求之与?抑与之与?”子贡曰:“夫子温、良、恭、俭、让以得之。夫子之求之也,其诸异乎人之求之与?”

    子贡曰:“夫子之文章,可得而闻也。夫子之言性与天道,不可得而闻也。”

    子贡曰:“纣之不善,不如是之甚也。是以君子恶居下流,天下之恶皆归焉。”
    
    子贡曰:“君子之过也,如日月之食焉。过也,人皆见之;更也,人皆仰之。”
    
    卫公孙朝问于子贡曰:“仲尼焉学?”子贡曰:“文武之道,未坠于地,在人。贤者识其大者,不贤者识其小者。莫不有文武之道焉。夫子焉不学?而亦何常师之有?”

    叔孙武叔语大夫于朝曰:“子贡贤于仲尼。”子服景伯以告子贡。子贡曰:“譬之宫墙,赐之墙也及肩,窥见室家之好。夫子之墙数仞,不得其门而入,不见宗庙之美,百官之富。得其门者或寡矣。夫子之云,不亦宜乎!”

    叔孙武叔毁仲尼。子贡曰:“无以为也!仲尼不可毁也。他人之贤者,丘陵也,犹可逾也。仲尼,日月也,无得而逾焉。人虽欲自绝,其何伤于日月乎?多见其不知量也。”

    陈子禽谓子贡曰:“子为恭也,仲尼岂贤于子乎?”子贡曰:“君子一言以为知,一言以为不知,言不可不慎也。夫子之不可及也,犹天之不可阶而升也。夫子之得邦家者,所谓立之斯立,道之斯行,绥之斯来,动之斯和。其生也荣,其死也哀,如之何其可及也?”

    \chapter{有子}

    有子曰:“其为人也孝弟,而好犯上者,鲜矣;不好犯上,而好作乱者,未之有也。君子务本,本立而道生。孝弟也者,其为仁之本与!”

    有子曰:“礼之用,和为贵。先王之道,斯为美,小大由之。有所不行,知和而和,不以礼节之,亦不可行也。”

    有子曰:“信近于义,言可复也。恭近于礼,远耻辱也。因不失其亲,亦可宗也。”

    哀公问于有若曰:“年饥,用不足,如之何?”有若对曰:“盍彻乎?”曰:“二,吾犹不足,如之何其彻也?”对曰:“百姓足,君孰与不足?百姓不足,君孰与足?”

    \chapter{子路}

    子路有闻,未之能行,惟恐有闻。

    子路无宿诺。

    \chapter{颜渊}

    颜渊喟然叹曰:“仰之弥高,钻之弥坚。瞻之在前,忽焉在后。夫子循循然善诱人,博我以文,约我以礼,欲罢不能。既竭吾才,如有所立卓尔。虽欲从之,末由也已。”

    \chapter{伯鱼}

    陈亢问于伯鱼曰:“子亦有异闻乎?”对曰:“未也。尝独立,鲤趋而过庭。曰:‘学诗乎?’对曰:‘未也。’‘不学诗,无以言。’鲤退而学诗。他日,又独立,鲤趋而过庭。曰:‘学礼乎?’对曰:‘未也。’‘不学礼,无以立。’鲤退而学礼。闻斯二者。”陈亢退而喜曰:“问一得三,闻诗,闻礼,又闻君子之远其子也。”

\end{document}
