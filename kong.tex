\documentclass[a5paper]{ctexbook}
\usepackage{geometry}
\usepackage{hyperref}
\geometry{a5paper}
\title{《论语》整理}
\author{安梯西登}
\date{}
\begin{document}
    \maketitle

    \tableofcontents

    \chapter{学}

    子曰:“学而时习之,不亦说乎?”

    子曰:“吾与回言终日,不违,如愚。退而省其私,亦足以发,回也不愚。”

    子曰:“学而不思则罔,思而不学则殆。”

    子曰:“攻乎异端,斯害也已。”

    \chapter{朋}
    
    子曰:“有朋自远方来,不亦乐乎?”

    \chapter{君子}
    
    子曰:“人不知而不愠,不亦君子乎?”

    子曰:“君子不重则不威,学则不固。主忠信。无友不如己者。过,则勿惮改。”

    子曰:“君子食无求饱,居无求安,敏于事而慎于言,就有道而正焉,可谓好学也已。”

    子曰:“君子不器。”
    
    子贡问君子。子曰:“先行其言而后从之。”
    
    子曰:“君子周而不比,小人比而不周。”

    子曰:“君子无所争。必也射乎!揖让而升,下而饮。其争也君子。”

    子曰:“君子之于天下也,无适也,无莫也,义之与比。”

    子曰:“君子怀德,小人怀土;君子怀刑,小人怀惠。”

    子曰:“君子喻于义,小人喻于利。”

    子曰:“君子欲讷于言而敏于行。”

    子谓子产:“有君子之道四焉:其行己也恭,其事上也敬,其养民也惠,其使民也义。”

    \chapter{仁}

    子曰:“巧言令色,鲜矣仁!”

    子曰:“人而不仁,如礼何?人而不仁,如乐何?”

    子夏问曰:“‘巧笑倩兮,美目盼兮,素以为绚兮。’何谓也?”子曰:“绘事后素。”曰:“礼后乎?”子曰:“起予者商也!始可与言《诗》已矣。”

    子谓《韶》,“尽美矣,又尽善也”。谓《武》,“尽美矣,未尽善也”。

    子曰:“里仁为美。择不处仁,焉得知?”

    子曰:“不仁者不可以久处约,不可以长处乐。仁者安仁,知者利仁。”

    子曰:“唯仁者能好人,能恶人。”

    子曰:“苟志于仁矣,无恶也。”

    子曰:“富与贵,是人之所欲也;不以其道得之,不处也。贫与贱,是人之所恶也;不以其道得之,不去也。君子去仁,恶乎成名?君子无终食之间违仁,造次必于是,颠沛必于是。”

    子曰:“人之过也,各于其党。观过,斯知仁矣。”

    子曰:“我未见好仁者,恶不仁者。好仁者,无以尚之;恶不仁者,其为仁矣,不使不仁者加乎其身。有能一日用其力于仁矣乎?我未见力不足者。盖有之矣,我未之见也。”

    子张问曰:“令尹子文三仕为令尹,无喜色;三已之,无愠色。旧令尹之政,必以告新令尹。何如?”子曰:“忠矣。”曰:“仁矣乎?”曰:“未知,焉得仁?”“崔子弑齐君,陈文子有马十乘,弃而违之。至于他邦,则曰:‘犹吾大夫崔子也。’违之。之一邦,则又曰:‘犹吾大夫崔子也。’违之。何如?”子曰:“清矣。”曰:“仁矣乎?”曰:“未知,焉得仁?”

    \chapter{诗}

    子曰:“诗三百,一言以蔽之,曰:‘思无邪。’”

    子曰:“《关雎》乐而不淫,哀而不伤。”

    \chapter{礼}

    子张问:“十世可知也?”子曰:“殷因于夏礼,所损益,可知也;周因于殷礼,所损益,可知也。其或继周者,虽百世,可知也。”

    子曰:“周监于二代,郁郁乎文哉!吾从周。”

    孔子谓季氏,“八佾舞于庭,是可忍也,孰不可忍也?”

    三家者以《雍》彻。子曰:“‘相维辟公,天子穆穆’,奚取于三家之堂?”

    林放问礼之本。子曰:“大哉问!礼,与其奢也,宁俭。丧,与其易也,宁戚。”

    子曰:“夷狄之有君,不如诸夏之亡也。”

    季氏旅于泰山。子谓冉有曰:“女弗能救与?”对曰:“不能。”子曰:“呜呼!曾谓泰山不如林放乎?”

    子曰:“夏礼,吾能言之,杞不足征也;殷礼,吾能言之,宋不足征也。文献不足故也。足,则吾能征之矣。”

    子曰:“禘自既灌而往者,吾不欲观之矣。”

    子入大庙,每事问。或曰:“孰谓鄹人之子知礼乎?入大庙,每事问。”子闻之,曰:“是礼也。”

    子贡欲去告朔之饩羊。子曰:“赐也!尔爱其羊,我爱其礼。”

    “然则管仲知礼乎?”曰:“邦君树塞门,管氏亦树塞门。邦君为两君之好,有反坫,管氏亦有反坫。管氏而知礼,孰不知礼?”

    \chapter{乐}

    子语鲁大师乐,曰:“乐其可知也:始作,翕如也;从之,纯如也,皦如也,绎如也,以成。”

    子谓《韶》,“尽美矣,又尽善也”。谓《武》,“尽美矣,未尽善也”。

    \chapter{鬼神}

    或问禘之说。子曰:“不知也。知其说者之于天下也,其如示诸斯乎!”指其掌。

    祭如在,祭神如神在。子曰:“吾不与祭,如不祭。”

    王孙贾问曰:“与其媚于奥,宁媚于灶,何谓也?”子曰:“不然。获罪于天,无所祷也。”

    \chapter{孝}

    子曰:“父在,观其志;父没,观其行;三年无改于父之道,可谓孝矣。”

    孟懿子问孝。子曰:“无违。”樊迟御,子告之曰:“孟孙问孝于我,我对曰,无违。”樊迟曰:“何谓也?”子曰:“生,事之以礼;死,葬之以礼,祭之以礼。”

    孟武伯问孝。子曰:“父母唯其疾之忧。”

    子游问孝。子曰:“今之孝者,是谓能养。至于犬马,皆能有养。不敬,何以别乎?”

    子夏问孝。子曰:“色难。有事,弟子服其劳;有酒食,先生馔,曾是以为孝乎?”

    子曰:“事父母几谏,见志不从,又敬不违,劳而不怨。”

    子曰:“父母在,不远游,游必有方。”

    子曰:“父母之年,不可不知也。一则以喜,一则以惧。”

    \chapter{信}

    子曰:“人而无信,不知其可也。大车无𫐐,小车无𫐄,其何以行之哉?”

    \chapter{俭}

    或曰:“管仲俭乎?”曰:“管氏有三归,官事不摄,焉得俭?”

    \chapter{治国}

    子曰:“道千乘之国,敬事而信,节用而爱人,使民以时。”

    子曰:“为政以德,譬如北辰居其所而众星共之。”

    子曰:“道之以政,齐之以刑,民免而无耻。道之以德,齐之以礼,有耻且格。”

    哀公问曰:“何为则民服?”孔子对曰:“举直错诸枉,则民服;举枉错诸直,则民不服。”

    季康子问:“使民敬、忠以劝,如之何?”子曰:“临之以庄,则敬;孝慈,则忠;举善而教不能,则劝。”

    子曰:“能以礼让为国乎?何有?不能以礼让为国,如礼何?”

    \chapter{修身}

    子曰:“弟子入则孝,出则弟,谨而信,泛爱众而亲仁。行有余力,则以学文。”

    子贡曰:“贫而无谄,富而无骄,何如?”子曰:“可也。未若贫而乐,富而好礼者也。”子贡曰:“《诗》云:‘如切如磋,如琢如磨。’其斯之谓与?”子曰:“赐也,始可与言《诗》已矣,告诸往而知来者。”

    \chapter{谓己}

    子曰:“吾十有五而志于学,三十而立,四十而不惑,五十而知天命,六十而耳顺,七十而从心所欲,不逾矩。”

    子曰:“十室之邑,必有忠信如丘者焉,不如丘之好学也。”

    \chapter{谓人}

    子在陈,曰:“归与!归与!吾党之小子狂简,斐然成章,不知所以裁之。”

    子谓公冶长:“可妻也。虽在缧绁之中,非其罪也。”以其子妻之。

    子谓南容:“邦有道,不废;邦无道,免于刑戮。”以其兄之子妻之。

    子谓子贱:“君子哉若人!鲁无君子者,斯焉取斯?”

    子贡问曰:“赐也何如?”子曰:“女,器也。”曰:“何器也?”曰:“瑚琏也。”

    子曰:“道不行,乘桴浮于海。从我者,其由与?”子路闻之喜。子曰:“由也好勇过我,无所取材。”

    孟武伯问:“子路仁乎?”子曰:“不知也。”又问。子曰:“由也,千乘之国,可使治其赋也,不知其仁也。”
    
    孟武伯问:“求也何如?”子曰:“求也,千室之邑,百乘之家,可使为之宰也,不知其仁也。”
    
    孟武伯问:“赤也何如?”子曰:“赤也,束带立于朝,可使与宾客言也,不知其仁也。”

    子谓子贡曰:“女与回也孰愈?”对曰:“赐也何敢望回?回也闻一以知十,赐也闻一以知二。”子曰:“弗如也,吾与女弗如也。”

    宰予昼寝。子曰:“朽木不可雕也,粪土之墙不可杇也。于予与何诛?”子曰:“始吾于人也,听其言而信其行;今吾于人也,听其言而观其行。于予与改是。”

    子曰:“吾未见刚者。”或对曰:“申枨。”子曰:“枨也欲,焉得刚?”

    子贡曰:“我不欲人之加诸我也,吾亦欲无加诸人。”子曰:“赐也,非尔所及也。”

    子曰:“晏平仲善与人交,久而敬之。”

    子曰:“臧文仲居蔡,山节藻棁,何如其知也?”

    子曰:“宁武子,邦有道,则知;邦无道,则愚。其知可及也,其愚不可及也。”

    子曰:“伯夷、叔齐不念旧恶,怨是用希。”

    子曰:“管仲之器小哉!”

    子曰:“孰谓微生高直?或乞醯焉,乞诸其邻而与之。”

    \chapter{杂}

    子曰:“不患人之不己知,患不知人也。”

    仪封人请见,曰:“君子之至于斯也,吾未尝不得见也。”从者见之。出曰:“二三子何患于丧乎?天下之无道也久矣,天将以夫子为木铎。”



    子曰:“视其所以,观其所由,察其所安。人焉廋哉?人焉廋哉?”

    子曰:“温故而知新,可以为师矣。”

    子曰:“由!诲女知之乎!知之为知之,不知为不知,是知也。”

    子张学干禄。子曰:“多闻阙疑,慎言其余,则寡尤;多见阙殆,慎行其余,则寡悔。言寡尤,行寡悔,禄在其中矣。”

    或谓孔子曰:“子奚不为政?”子曰:“《书》云:‘孝乎惟孝,友于兄弟,施于有政。’是亦为政,奚其为为政?”

    子曰:“非其鬼而祭之,谄也。见义不为,无勇也。”

    子曰:“射不主皮,为力不同科,古之道也。”

    子曰:“事君尽礼,人以为谄也。”

    定公问:“君使臣,臣事君,如之何?”孔子对曰:“君使臣以礼,臣事君以忠。”

    或曰:“雍也仁而不佞。”子曰:“焉用佞?御人以口给,屡憎于人。不知其仁,焉用佞?”

    哀公问社于宰我。宰我对曰:“夏后氏以松,殷人以柏,周人以栗,曰使民战栗。”子闻之,曰:“成事不说,遂事不谏,既往不咎。”

    子使漆雕开仕。对曰:“吾斯之未能信。”子说。
    
    子曰:“朝闻道,夕死可矣。”
    
    子曰:“士志于道,而耻恶衣恶食者,未足与议也。”

    子贡问曰:“孔文子何以谓之‘文’也?”子曰:“敏而好学,不耻下问,是以谓之‘文’也。”

    子曰:“不患无位,患所以立。不患莫己知,求为可知也。”

    子曰:“见贤思齐焉,见不贤而内自省也。”

    子曰:“古者言之不出,耻躬之不逮也。”

    季文子三思而后行。子闻之,曰:“再,斯可矣。”
    
    子曰:“以约失之者鲜矣。”

    子曰:“德不孤,必有邻。”

    子曰:“巧言、令色、足恭,左丘明耻之,丘亦耻之。匿怨而友其人,左丘明耻之,丘亦耻之。”

    颜渊、季路侍。子曰:“盍各言尔志?”子路曰:“愿车马衣轻裘与朋友共敝之而无憾。”颜渊曰:“愿无伐善,无施劳。”子路曰:“愿闻子之志。”子曰:“老者安之,朋友信之,少者怀之。”

    子曰:“已矣乎!吾未见能见其过而内自讼者也。”

    

    \chapter{子张}

    子张曰:“士见危致命,见得思义,祭思敬,丧思哀,其可已矣。”
    
    子张曰:“执德不弘,信道不笃,焉能为有?焉能为亡?”
    
    子夏之门人问交于子张。子张曰:“子夏云何?”对曰:“子夏曰:‘可者与之,其不可者拒之。’”子张曰:“异乎吾所闻。君子尊贤而容众,嘉善而矜不能。我之大贤与,于人何所不容?我之不贤与,人将拒我,如之何其拒人也?”

    \chapter{子夏}

    子夏曰:“贤贤易色;事父母,能竭其力;事君,能致其身;与朋友交,言而有信。虽曰未学,吾必谓之学矣。”

    樊迟退,见子夏曰:“乡也吾见于夫子而问知,子曰:‘举直错诸枉,能使枉者直。’何谓也?”子夏曰:“富哉言乎!舜有天下,选于众,举皋陶,不仁者远矣。汤有天下,选于众,举伊尹,不仁者远矣。”

    子夏曰:“虽小道,必有可观者焉,致远恐泥,是以君子不为也。”
    
    子夏曰:“日知其所亡,月无忘其所能,可谓好学也已矣。”
    
    子夏曰:“博学而笃志,切问而近思,仁在其中矣。”

    子夏曰:“百工居肆以成其事,君子学以致其道。”
    
    子夏曰:“小人之过也必文。”
    
    子夏曰:“君子有三变:望之俨然,即之也温,听其言也厉。”

    子夏曰:“君子信而后劳其民;未信,则以为厉己也。信而后谏;未信,则以为谤己也。”
    
    子夏曰:“大德不逾闲,小德出入可也。”
    
    子游曰:“子夏之门人小子,当洒扫应对进退,则可矣,抑末也。本之则无,如之何?”子夏闻之,曰:“噫!言游过矣!君子之道,孰先传焉?孰后倦焉?譬诸草木,区以别矣。君子之道,焉可诬也?有始有卒者,其惟圣人乎!”

    子夏曰:“仕而优则学,学而优则仕。”

    \chapter{子游}

    子游曰:“事君数,斯辱矣。朋友数,斯疏矣。”

    子游曰:“丧致乎哀而止。”
    
    子游曰:“吾友张也为难能也,然而未仁。”

    \chapter{曾子}

    曾子曰:“吾日三省吾身:为人谋而不忠乎?与朋友交而不信乎?传不习乎?”

    曾子曰:“慎终追远,民德归厚矣。”

    子曰:“参乎!吾道一以贯之。”曾子曰:“唯。”子出,门人问曰:“何谓也?”曾子曰:“夫子之道,忠恕而已矣。”

    曾子有疾,召门弟子曰:“启予足!启予手!《诗》云:‘战战兢兢,如临深渊,如履薄冰。’而今而后,吾知免夫!小子!”

    曾子有疾,孟敬子问之。曾子言曰:“鸟之将死,其鸣也哀。人之将死,其言也善。君子所贵乎道者三:动容貌,斯远暴慢矣;正颜色,斯近信矣;出辞气,斯远鄙倍矣。笾豆之事,则有司存。”

    曾子曰:“以能问于不能,以多问于寡;有若无,实若虚,犯而不校。昔者吾友尝从事于斯矣。”
    
    曾子曰:“可以托六尺之孤,可以寄百里之命,临大节而不可夺也,君子人与?君子人也。”
    
    曾子曰:“士不可以不弘毅,任重而道远。仁以为己任,不亦重乎?死而后已,不亦远乎?”

    曾子曰:“君子以文会友,以友辅仁。”
    
    曾子曰:“君子思不出其位。”

    曾子曰:“堂堂乎张也,难于并为仁矣。”
    
    曾子曰:“吾闻诸夫子:人未有自致者也,必也亲丧乎!”
    
    曾子曰:“吾闻诸夫子:孟庄子之孝也,其他可能也,其不改父之臣与父之政,是难能也。”

    孟氏使阳肤为士师,问于曾子。曾子曰:“上失其道,民散久矣。如得其情,则哀矜而勿喜!”

    \chapter{子贡}

    子禽问于子贡曰:“夫子至于是邦也,必闻其政。求之与?抑与之与?”子贡曰:“夫子温、良、恭、俭、让以得之。夫子之求之也,其诸异乎人之求之与?”

    子贡曰:“夫子之文章,可得而闻也。夫子之言性与天道,不可得而闻也。”

    子贡曰:“纣之不善,不如是之甚也。是以君子恶居下流,天下之恶皆归焉。”
    
    子贡曰:“君子之过也,如日月之食焉。过也,人皆见之;更也,人皆仰之。”
    
    卫公孙朝问于子贡曰:“仲尼焉学?”子贡曰:“文武之道,未坠于地,在人。贤者识其大者,不贤者识其小者。莫不有文武之道焉。夫子焉不学?而亦何常师之有?”

    叔孙武叔语大夫于朝曰:“子贡贤于仲尼。”子服景伯以告子贡。子贡曰:“譬之宫墙,赐之墙也及肩,窥见室家之好。夫子之墙数仞,不得其门而入,不见宗庙之美,百官之富。得其门者或寡矣。夫子之云,不亦宜乎!”

    叔孙武叔毁仲尼。子贡曰:“无以为也!仲尼不可毁也。他人之贤者,丘陵也,犹可逾也。仲尼,日月也,无得而逾焉。人虽欲自绝,其何伤于日月乎?多见其不知量也。”

    陈子禽谓子贡曰:“子为恭也,仲尼岂贤于子乎?”子贡曰:“君子一言以为知,一言以为不知,言不可不慎也。夫子之不可及也,犹天之不可阶而升也。夫子之得邦家者,所谓立之斯立,道之斯行,绥之斯来,动之斯和。其生也荣,其死也哀,如之何其可及也?”

    \chapter{有子}

    有子曰:“其为人也孝弟,而好犯上者,鲜矣;不好犯上,而好作乱者,未之有也。君子务本,本立而道生。孝弟也者,其为仁之本与!”

    有子曰:“礼之用,和为贵。先王之道,斯为美,小大由之。有所不行,知和而和,不以礼节之,亦不可行也。”

    有子曰:“信近于义,言可复也。恭近于礼,远耻辱也。因不失其亲,亦可宗也。”

    哀公问于有若曰:“年饥,用不足,如之何?”有若对曰:“盍彻乎?”曰:“二,吾犹不足,如之何其彻也?”对曰:“百姓足,君孰与不足?百姓不足,君孰与足?”

    \chapter{子路}

    子路有闻,未之能行,惟恐有闻。

    子路无宿诺。

    \chapter{颜渊}

    颜渊喟然叹曰:“仰之弥高,钻之弥坚。瞻之在前,忽焉在后。夫子循循然善诱人,博我以文,约我以礼,欲罢不能。既竭吾才,如有所立卓尔。虽欲从之,末由也已。”

    \chapter{伯鱼}

    陈亢问于伯鱼曰:“子亦有异闻乎?”对曰:“未也。尝独立,鲤趋而过庭。曰:‘学诗乎?’对曰:‘未也。’‘不学诗,无以言。’鲤退而学诗。他日,又独立,鲤趋而过庭。曰:‘学礼乎?’对曰:‘未也。’‘不学礼,无以立。’鲤退而学礼。闻斯二者。”陈亢退而喜曰:“问一得三,闻诗,闻礼,又闻君子之远其子也。”

\end{document}
